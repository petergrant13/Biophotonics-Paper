\documentclass{article}

\usepackage{braket}
\usepackage{amsmath}
\usepackage{amsfonts}
\usepackage{amssymb}
\usepackage{xcolor}
\usepackage{chngcntr}
\usepackage{graphicx}
\usepackage{tikz}
\usepackage{pgfplots}
\usepackage{siunitx}
\usepackage{filecontents}
\usepackage{cite}
\usepackage{hyperref}
\usepackage{titlesec}
\pgfplotsset{width=15cm,compat=1.9}

\newcommand{\R}{\mathbb{R}}
\newcommand{\C}{\mathbb{C}}
\newcommand{\p}{\partial}
\renewcommand{\L}{\mathcal{L}}
\renewcommand{\H}{\mathcal{H}}

\long\def\/*#1*/{} %used to comment out multiple lines

\/*
     This text is commented out
*/

\counterwithin*{equation}{section}
\counterwithin*{equation}{subsection}

\setlength{\topmargin}{-.5in}
\setlength{\textheight}{9in}
\setlength{\oddsidemargin}{.125in}
\setlength{\textwidth}{6.25in}

\begin{document}
\begin{titlepage}
    \begin{center}
        \vspace*{2cm}
 
        \huge
        \textbf{ECE 434 Biophotonics Project}
     
        \vspace{1.5cm}
        \Large
        \textbf{Peter Grant, Evan Peters\\
        V00948581, V00954410}
 
        \vfill
             
        A Project Report on optical trapping\\
        For Dr. Tao Lu     

        \vspace{0.8cm}
      

             
        Department of Electrical and Computer Engineering\\
        University of Victoria\\
        April 8, 2024
             
    \end{center}
\end{titlepage}


\tableofcontents
\newpage

\addcontentsline{toc}{section}{Introduction}
\section*{Introduction}

TODO

\addcontentsline{toc}{section}{Theory of Lasers}
\section*{Theory of Lasers}

To understand the theory of laser tweezers, it is necessary to understand the theory of lasers. This section will cover the basic theory of lasers, including the theory of light-matter interactions, perturbation theory, and the lasing principle.

\addcontentsline{toc}{subsection}{Light-Matter Interactions}
\subsection*{Light-Matter Interactions}

TODO


\addcontentsline{toc}{subsection}{Perturbation Theory}
\subsection*{Perturbation Theory}


There are only a handful of potentials for which the Schr\"odinger equation can be solved exactly. For the vast majority of potentials, we must resort to perturbation theory. The idea is to start with a simple potential for which the Schr\"odinger equation can be solved exactly, and then add a small perturbation to the potential. The Schr\"odinger equation can then be solved iteratively, with each term in the series representing a higher order correction to the wavefunction. 

In general, time-independent perturbation theory is used when the Hamiltonian can be written as the sum of two terms, $\H = \H^0 + \H'$, where $\H^0$ is the Hamiltonian for a system for which the Schr\"odinger equation can be solved exactly, and $\H'$ is a small perturbation.

For lasers, time-dependent perturbation theory is necessary. For example, the perturbation could be a sinusoidal electric field, which is used to model the interaction between light and matter. For time-dependent perturbation theory of a two-level system with two states that are eigenstates of the unperturbed Hamiltonian,

\[ \hat{\H}^0\psi_a = E_a\psi_a,\quad \hat{\H}^0\psi_b = E_b\psi_b \]

The time evolution of the system is thus 
\[ \Psi(t) = c_a\psi_ae^{-iE_at/\hbar} + c_b\psi_be^{-iE_bt/\hbar} \]

Then, the perturbation is "turned on". The result is that the coefficients now depend on time 
\[ \Psi(t) = c_a(t)\psi_ae^{-iE_at/\hbar} + c_b(t)\psi_be^{-iE_bt/\hbar} \]

The desired quantities are $c_a(t)$ and $c_b(t)$. These can be found by solving the time-dependent Schr\"odinger equation with the perturbed Hamiltonian. 
\[ \hat{\H}\Psi = i\hbar\frac{\p\Psi}{\p t},\ \hat{\H} = \hat{\H}^0 + \hat{\H}'(t) \]

Solving this equation is not trivial, and is certainly beyond the scope of this paper. Please refer to the references for more information. However, because the states are orthogonal, the trick of taking the inner product of each state can be used. The essential result is 
\begin{equation}
     \dot{c}_a = -\frac{i}{\hbar}\H'_{ab}e^{-i\omega_0t}c_b,\quad \dot{c}_b = -\frac{i}{\hbar}\H'_{ba}e^{i\omega_0t}c_a 
\end{equation}

Where 
\[ \omega_0 \equiv \frac{E_b - E_a}{\hbar} \]

So, this says that the probability of transitioning from state $a$ to state $b$ is proportional to the matrix element of the perturbation between the two states.

Thus far, the coefficients c are exact. However, to calculate them exactly requires infinitely higher order perturbations. Generally, the first order perturbation is sufficient, and will be examined for the rest of the paper.

\addcontentsline{toc}{subsection}{Sinusoidal Perturbations}
\subsection*{Sinusoidal Perturbations}

Since electromagnetic radiation is sinusoidal, it is extremely useful to look at sinusoidal perturbations. Consider the Hamiltonian
\[ \hat{\H}'(\vec{r},t) = V(\vec{r})\cos(\omega t) \]

This will have matrix element 
\[ \H'_{ab} = V_{ab}\cos(\omega t) \]

Where 
\[ V_{ab} = \braket{\psi_a|V(\vec{r})|\psi_b} \]

Plugging this result into (1) gives
\[ c_b(t) \approx -\frac{i}{\hbar}V_{ba}\int_0^tdt'\ \cos(\omega t')e^{i\omega_0t''}  \]
\[ = -\frac{V_{ba}}{2\hbar}\left[ \frac{e^{i(\omega_0+\omega)t}-1}{\omega_0+\omega} + \frac{e^{i(\omega_0+\omega)t}-1}{\omega_0-\omega} \right] \]

It is obvious that for frequencies far from the resonance frequency, the transition probability will be quite small. So, it is useful to only look at frequencies close to the resonant frequencies, and neglect the first term. This means that the transition probability for a sinusoidal perturbation is given by 
\[ P_{a\to b}(t) = |c_b(t)|^2 \approx \frac{|V_{ab}|^2}{\hbar}\frac{\sin^2\left[(\omega_0-\omega)t/2\right]}{(\omega_0-\omega)^2} \]

So, the incident electromagnetic wave will need to have a frequency close to the resonant frequency in order to have a significant transition probability.

Further theory would look at incoherent electromagnetic radiation from all directions, then use Boltzmann statistics and Planck's blackbody formula to derive Einstein's famous A and B coefficients to get the rates of transmissions. Then, one can get the lifetime and selection rules of the states, along with Fermi's golden rule to engineer the laser.


\addcontentsline{toc}{subsection}{Lasing Principle}
\subsection*{Lasing Principle}

This has been covered in depth throughout the course. 




\addcontentsline{toc}{section}{Theory of Laser Tweezers}
\section*{Theory of Laser Tweezers}


\addcontentsline{toc}{subsection}{The Ray Optics Model}
\subsection*{The Ray Optics Model}

TODO


\addcontentsline{toc}{subsection}{The Electric Dipole Model}
\subsection*{The Electric Dipole Model}

For small objects, the ray optics model no longer works. So, more theory is required. The object can be modeled as an electrical dipole to approximate the photon and particle interactions. The force acting on a single point charge placed in a magnetic field is known as the Lorentz force \cite{UToronto}

\[ \mathbf{F} = (\mathbf{p}\ \cdot\mathbf{\nabla})\mathbf{E} + \frac{d\mathbf{p}}{dt}\times\mathbf{B} \]

Here, $\mathbf{p}$ is the dipole moment of the object, $\mathbf{E}$ is the electric field, and $\mathbf{B}$ is the magnetic field. The first term is the force due to the electric field, and the second term is the force due to the magnetic field.

Through the use of polarizability, $\alpha$, the dipole can be eliminated.

\[ \mathbf{F} = \alpha\left[ (\mathbf{E}\cdot\mathbf{\nabla})\mathbf{E} + \frac{d\mathbf{E}}{dt}\times\mathbf{B} \right]  \]
\[ \mathbf{F} = \alpha\left[ \frac{1}{2}\mathbf{\nabla}\mathbf{E}^2 + \frac{d}{dt}(\mathbf{E}\times\mathbf{B}) \right]  \]

Then, the term on the right is the time derivative of the Poynting vector. For a CW laser, this term is simply zero
\[ \mathbf{F} = \frac{1}{2}\alpha\nabla E^2 \]



\[ \mathbf{F}_\text{scat}(\mathbf{r}) = \frac{k^4\alpha^2}{6\pi cn^3\varepsilon_0^2}\mathbf{I}(\mathbf{r})\hat{z} = \frac{8\pi n_0k^4a^6}{3c}\left( \frac{m^2-1}{m^2+2} \right)^2\mathbf{I}(\mathbf{r})\hat{z}  \]

\addcontentsline{toc}{subsection}{Harmonic Potential Approximation}
\subsection*{Harmonic Potential Approximation}

\[ \mathbf{\nabla E}_\text{AC Stark} = \frac{3\pi c^2\Gamma\mu}{2\omega_0^3\delta}\mathbf{I(r,z)}  \]

\[ I(r,z) = I_0 \left( \frac{\omega_0}{\omega(z)} \right)^2\ e^{\displaystyle -\frac{2r^2}{\omega^2(z)}}  \]

\[ \omega(z) = \omega_0 \sqrt{1 + \left( \frac{z}{Z_R} \right)^2} \]

\[ Z_R = \frac{\pi\omega_0^2}{\lambda}  \]

\[ P_0 = \frac{1}{2} \pi I_0\omega_0^2 \]

\[ \frac{1}{2!}\frac{\p^2I}{\p z^2} \Bigg\vert_{r,z=0} z^2 = \frac{2P_0\lambda^2}{\pi^3\omega_0^6}z^2 = \frac{1}{2}m\omega_z^2z^2  \]

\[ \frac{1}{2!}\frac{\p^2I}{\p r^2} \Bigg\vert_{r,z=0} r^2 = \frac{4P_0}{\pi\omega_0^4}r^2 = \frac{1}{2}m\omega_r^2r^2  \]

\[ \omega_r = \sqrt{\frac{8P_0}{\pi m\omega_0^4}} \]
\[ \omega_z = \sqrt{\frac{4P_0\lambda^2}{m\pi^3\omega_0^6}} \]
\[ \frac{\omega_r}{\omega_z} = \sqrt{2}\frac{\omega_0\pi}{\lambda}  \]





\addcontentsline{toc}{section}{Applications}
\section*{Applications}





\newpage
\bibliography{Term_Paper}
\bibliographystyle{ieeetr}

% https://www.physics.utoronto.ca/apl/opt/opt.pdf
% https://home.uni-leipzig.de/pwm/web/?section=introduction&page=opticaltraps
% https://www.sciencedirect.com/science/article/pii/S0091679X06820066?via%3Dihub




\end{document}