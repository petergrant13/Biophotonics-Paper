\documentclass{article}

\usepackage{braket}
\usepackage{amsmath}
\usepackage{amsfonts}
\usepackage{amssymb}
\usepackage{xcolor}
\usepackage{chngcntr}
\usepackage{graphicx}
\usepackage{tikz}
\usepackage{pgfplots}
\pgfplotsset{width=15cm,compat=1.9}

\newcommand{\R}{\mathbb{R}}
\newcommand{\C}{\mathbb{C}}
\newcommand{\p}{\partial}
\renewcommand{\L}{\mathcal{L}}
\renewcommand{\H}{\mathcal{H}}

\long\def\/*#1*/{} %used to comment out multiple lines

\/*
     This text is commented out
*/

\counterwithin*{equation}{section}
\counterwithin*{equation}{subsection}

\setlength{\topmargin}{-.5in}
\setlength{\textheight}{9in}
\setlength{\oddsidemargin}{.125in}
\setlength{\textwidth}{6.25in}

\begin{document}
\begin{titlepage}
    \begin{center}
        \vspace*{2cm}
 
        \huge
        \textbf{ECE 434 Biophotonics Project}
     
        \vspace{1.5cm}
        \Large
        \textbf{Peter Grant, Evan Peters\\
        V00948581, V00954410}
 
        \vfill
             
        A Project Report on optical trapping\\
        For Dr. Tao Lu     

        \vspace{0.8cm}
      

             
        Department of Electrical and Computer Engineering\\
        University of Victoria\\
        April 8, 2024
             
    \end{center}
\end{titlepage}


\tableofcontents
\newpage

\addcontentsline{toc}{section}{Theory}
\section*{Theory}

\addcontentsline{toc}{subsection}{The Ray Optics Model}
\subsection*{The Ray Optics Model}

TODO


\addcontentsline{toc}{subsection}{The Electric Dipole Model}
\subsection*{The Electric Dipole Model}

\[ \mathbf{F} = (\mathbf{p}\ \cdot\mathbf{\nabla})\mathbf{E} + \frac{d\mathbf{p}}{dt}\times\mathbf{B} \]


\[ \mathbf{F} = \alpha\left[ (\mathbf{E}\cdot\mathbf{\nabla})\mathbf{E} + \frac{d\mathbf{E}}{dt}\times\mathbf{B} \right]  \]
\[ \mathbf{F} = \alpha\left[ \frac{1}{2}\mathbf{\nabla}\mathbf{E}^2 + \frac{d}{dt}(\mathbf{E}\times\mathbf{B}) \right]  \]

\[ \mathbf{F} = \frac{1}{2}\alpha\nabla E^2 \]

\[ \mathbf{F}_\text{scat}(\mathbf{r}) = \frac{k^4\alpha^2}{6\pi cn^3\varepsilon_0^2}\mathbf{I}(\mathbf{r})\hat{z} = \frac{8\pi n_0k^4a^6}{3c}\left( \frac{m^2-1}{m^2+2} \right)^2\mathbf{I}(\mathbf{r})\hat{z}  \]

\addcontentsline{toc}{subsection}{Harmonic Potential Approximation}
\subsection*{Harmonic Potential Approximation}

\[ \mathbf{\nabla E}_\text{AC Stark} = \frac{3\pi c^2\Gamma\mu}{2\omega_0^3\delta}\mathbf{I(r,z)}  \]

\[ I(r,z) = I_0 \left( \frac{\omega_0}{\omega(z)} \right)^2\ e^{\displaystyle -\frac{2r^2}{\omega^2(z)}}  \]

\[ \omega(z) = \omega_0 \sqrt{1 + \left( \frac{z}{Z_R} \right)^2} \]

\[ Z_R = \frac{\pi\omega_0^2}{\lambda}  \]

\[ P_0 = \frac{1}{2} \pi I_0\omega_0^2 \]

\[ \frac{1}{2!}\frac{\p^2I}{\p z^2} \Bigg\vert_{r,z=0} z^2 = \frac{2P_0\lambda^2}{\pi^3\omega_0^6}z^2 = \frac{1}{2}m\omega_z^2z^2  \]

\[ \frac{1}{2!}\frac{\p^2I}{\p r^2} \Bigg\vert_{r,z=0} r^2 = \frac{4P_0}{\pi\omega_0^4}r^2 = \frac{1}{2}m\omega_r^2r^2  \]

\[ \omega_r = \sqrt{\frac{8P_0}{\pi m\omega_0^4}} \]
\[ \omega_z = \sqrt{\frac{4P_0\lambda^2}{m\pi^3\omega_0^6}} \]
\[ \frac{\omega_r}{\omega_z} = \sqrt{2}\frac{\omega_0\pi}{\lambda}  \]



Want to go over time dependent perturbation theory


Then add sinusoidal perturbations


Then derive Einstein's A and B coeffs


Ultimately get Fermi's golden rule and lasers

\addcontentsline{toc}{section}{Applications}
\section*{Applications}


\newpage
\section*{References}

% https://www.physics.utoronto.ca/apl/opt/opt.pdf



\end{document}